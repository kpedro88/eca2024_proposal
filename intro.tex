\section{Overview}\label{sec:intro}

\subsection{Program and Significance}\label{subsec:program}

This proposal addresses two important, outstanding topics in particle physics at the energy frontier:
the nature of dark matter (DM) and the computational challenge of detector simulation.
The new P5 report~\cite{P5:2023} has reaffirmed that determining the nature of DM,
as well as searching for direct evidence of new particles, are key science drivers for the field.
Further, software and computing are noted as key enablers and areas for investment.
These topics are crucial for the Large Hadron Collider (LHC) program,
including the upcoming high-luminosity upgrade (HL-LHC).

In particular, this proposal explores the hypothesis that dark matter, much like ordinary matter,
consists of composite particles bound together by a strong force, analogous to standard model (SM) quantum chromodynamics (QCD).
This is predicted by ``hidden valley'' models~\cite{Strassler:2006im}, which postulate a dark sector with a new, confining force, called ``dark QCD''.
\textbf{Composite, strongly coupled DM can only be accessed at colliders.}
Its interactions with ordinary matter are expected to be highly suppressed,
and therefore it would evade direct detection experiments~\cite{Cohen:2017pzm}.
Its abundance arises from an asymmetry mechanism, so it would similarly evade annihilation-based experiments~\cite{Petraki:2013wwa}.
Nevertheless, such models can satisfy observed relic density constraints~\cite{Beauchesne:2018myj,Beauchesne:2019ato}.
These models include numerous parameters with unknown values and theoretical uncertainties.
Therefore, we will increase model independence in every step of the search compared to previous efforts,
in particular by employing unsupervised artificial intelligence (AI).
\textbf{Now is the most opportune time to conduct this search}:
LHC Run 3 will provide more and higher-energy collisions with new triggers,
and the results will inform preparations for the HL-LHC.

Accurate detector simulation is one of the cornerstones of HEP, crucial to design searches for new physics.
Because of the wide range of possible signal models,
as well as the potentially large backgrounds from high-cross section processes such as SM QCD multijet production,
the dark QCD program will require substantial quantities of simulated events.
The current approach, modeling each simulated particle's interactions with the detector material, is too computationally intensive to deliver enough events.
In this way, the search for composite DM presages a challenge soon to be faced by the entire field at the HL-LHC.
With data rates increasing by a factor of 10 along with growth in event complexity, no analysis will have a sufficient amount of simulation.
\textbf{Generative AI is well suited to address this challenge, dramatically accelerating the computation of detector simulation while retaining quality.}
AI-based simulation can naturally take advantage of GPUs and other coprocessors, such as those at high performance computing (HPC) centers.
Using inference as a service, these new techniques will be integrated seamlessly into the existing software with maximal flexibility to use coprocessors efficiently.

The PI is a recognized leader in strongly coupled dark matter and both unsupervised and generative AI.
His leadership is supplemented by broad expertise in collider searches and scientific computing.
The program proposed here will advance multiple elements of the DOE HEP mission: understanding dark matter and improving capabilities in detector simulation.
The techniques developed to execute this program will be broadly useful in many fields:
model-independent searches using unsupervised AI and fast, accurate simulation using generative AI are of interest
not only in collider physics, but also in neutrino physics and astrophysics, among others.
These benefits will even extend to future colliders, beyond the HL-LHC.

\subsection{Team}\label{subsec:team}

With the early career funding, the PI will assemble a team with the necessary expertise to carry out this program.
The PI and a new postdoctoral research associate (RA) will co-lead both the dark QCD search and the development of AI-based detector simulation.
Fermilab provides a unique AI associate (AIA) program, which hires post-baccalaureate researchers with expertise in AI and computing.
Two AI associates (AIA\#1 and AIA\#2) will join the team at different stages to deliver the AI algorithms and to implement the computing for AI-based simulation, respectively.

The PI will leverage his established collaborations at Fermilab to provide unparalleled and highly relevant resources and knowledge to the team.
Fermilab theorists work on dark sector modeling and phenomenology.
Fermilab hosts one of the largest detector simulation groups and the CMS experiment core software developers, with whom the PI works closely.
Fermilab also has a diverse but tight-knit group of AI experts working on colliders, neutrinos, astrophysics, and computing.
The PI's position and connections at Fermilab, therefore, are multipliers for the DOE's early career investment, strengthening the proposal.

\subsection{Collaborators}\label{subsec:collab}

The PI started and leads the CMS Dark QCD team, which includes several Fermilab RAs and scientists.
There are also numerous university personnel via the LHC Physics Center (LPC)~\cite{LPC}, hosted at Fermilab, from various institutes:
Boston, MIT, Colorado, Maryland, Puerto Rico, Rochester, and Tennessee.
The team additionally collaborates with international partners, associated with CERN, from ETH Zurich, KIT, and University of Zurich.
In particular, we benefit from the contributions of top-tier graduate students and RAs, including several LPC Graduate Scholars, LPC Distinguished Researchers, and a Fermilab Lederman Fellow.

As convener of the CMS ML4Sim group, the PI collaborates closely with international colleagues from institutes such as DESY, University of Hamburg, and National Taiwan University.
The PI developed inference as a service for experiment software frameworks as part of the Fast Machine Learning Lab~\cite{FML},
a collective that includes Fermilab researchers (above)
and others from CERN, MIT, Purdue, San Diego, Colorado, Urbana-Champaign, and Washington.
The Fast ML Lab has strong ties to industry, including Nvidia, Microsoft, and Graphcore, and the PI in particular has extensive experience working with industry engineers.
This diverse group, which the PI can most effectively access as a fellow lab scientist,
provides unparalleled knowledge and experience to support the proposal's technical and ML-related goals.
