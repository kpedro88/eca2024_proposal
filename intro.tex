\section{Introduction}\label{sec:intro}

This proposal addresses two of the most important outstanding topics in particle physics at the energy frontier:
the nature of dark matter (DM) and the computational challenge of detector simulation.
The recent P5 report~\cite{P5:2023} has reaffirmed determining the nature of DM,
as well as searching for direct evidence of new particles, as key science drivers for the field.
Further, software and computing are noted as key enablers and areas for investment.
These topics are crucial for the Large Hadron Collider (LHC) physics program,
including the upcoming high-luminosity upgrade (HL-LHC).

In particular, this proposal explores the hypothesis that dark matter, much like ordinary matter,
consists of composite particles bound together by a strong force, analogous to standard model (SM) quantum chromodynamics (QCD).
This is predicted by ``hidden valley'' models, which postulate a dark sector with a new, confining force, called ``dark QCD''.
Such composite, or strongly coupled, dark matter is expected to have highly suppressed interactions with ordinary matter
and therefore would evade detection at direct and annihilation-based dark matter experiments~\cite{Cohen:2017pzm,Petraki:2013wwa}.
However, evidence of dark QCD may be found at colliders, illustrating the complementarity between the different prongs of the high energy physics (HEP) program.
Collider production would result in novel phenomenological signatures, such as ``semivisible jets'' containing a mixture of visible and invisible particles.
Because these models include numerous parameters with unknown values and theoretical uncertainties, the most effective search strategy will cover a broad range of similar phenomena.
This will be accomplished through unsupervised artificial intelligence (AI), which learns to distinguish known standard model (SM) processes from anomalies
without relying on a specific model of new physics.
Unsupervised AI will be employed both to select events at the trigger level and to identify new phenomena later in the analysis.

Accurate detector simulation is one of the cornerstones of HEP, crucial to design searches for new physics.
Because of the wide range of possible signal models,
as well as the potentially large backgrounds from high-cross section processes such as SM QCD multijet production,
this program will require substantial quantities of simulated events.
The current approach, modeling each simulated particle's interactions with the detector material, is too computationally intensive to deliver enough events.
In this way, the search for composite dark matter presages a challenge soon to be faced by the entire field at the HL-LHC.
With data rates increasing by a factor of 10 and event complexity growing similarly, no analysis will have a sufficient amount of simulation.
New techniques in generative AI are well suited to address this challenge, dramatically accelerating the computation of detector simulation while retaining quality.
AI-based simulation can naturally take advantage of GPUs and other coprocessors, such as those at high performance computing (HPC) centers.
Using inference as a service, these new techniques will be implemented with minimal disruption to the existing software and maximal flexibility to use coprocessors efficiently.

The PI is a recognized leader in strongly coupled dark matter and both unsupervised and generative AI.
His leadership is supplemented by broad expertise in collider searches for new physics and scientific computing.
The program proposed here will advance multiple elements of the DOE HEP mission: understanding dark matter and improving capabilities in detector simulation.
The benefits will extend to the entire field and even into the future, beyond the HL-LHC, providing physics motivation and critical techniques for future colliders.
Success will be achieved by leveraging the PI's established collaborations based at Fermilab, which provides unparalleled and highly relevant resources and knowledge.
