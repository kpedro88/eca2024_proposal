\section{Introduction}\label{sec:intro}

This proposal addresses two of the most important outstanding topics in particle physics at the energy frontier:
the nature of dark matter (DM) and the computational challenge of detector simulation.
The recent P5 report~\cite{P5:2023} has reaffirmed determining the nature of DM,
as well as searching for direct evidence of new particles, as key science drivers for the field.
Further, software and computing are noted as key enablers and areas for investment.
These topics are crucial for the Large Hadron Collider (LHC) physics program,
including the upcoming high-luminosity upgrade (HL-LHC).

In particular, this proposal explores the hypothesis that dark matter, much like ordinary matter,
consists of composite particles bound together by a strong force, analogous to standard model (SM) quantum chromodynamics (QCD).
This is predicted by ``hidden valley'' models~\cite{Strassler:2006im}, which postulate a dark sector with a new, confining force, called ``dark QCD''.
This composite, strongly coupled dark matter is expected to have highly suppressed interactions with ordinary matter
and therefore would evade detection at direct and annihilation-based dark matter experiments~\cite{Cohen:2017pzm,Petraki:2013wwa}.
However, evidence of dark QCD may be found at colliders, illustrating the complementarity between the different prongs of the high energy physics (HEP) program.
Collider production would result in several varieties of novel phenomenological signatures,
and because these models include numerous parameters with unknown values and theoretical uncertainties, the most effective search strategy will cover a broad range of such phenomena.
This will be accomplished through unsupervised artificial intelligence (AI), which learns to distinguish known standard model (SM) processes from anomalies
without relying on a specific model of new physics.
Unsupervised AI will be employed both to select events at the trigger level and to identify new phenomena later in the analysis.

Accurate detector simulation is one of the cornerstones of HEP, crucial to design searches for new physics.
Because of the wide range of possible signal models,
as well as the potentially large backgrounds from high-cross section processes such as SM QCD multijet production,
this program will require substantial quantities of simulated events.
The current approach, modeling each simulated particle's interactions with the detector material, is too computationally intensive to deliver enough events.
In this way, the search for composite dark matter presages a challenge soon to be faced by the entire field at the HL-LHC.
With data rates increasing by a factor of 10 and event complexity growing similarly, no analysis will have a sufficient amount of simulation.
New techniques in generative AI are well suited to address this challenge, dramatically accelerating the computation of detector simulation while retaining quality.
AI-based simulation can naturally take advantage of GPUs and other coprocessors, such as those at high performance computing (HPC) centers.
Using inference as a service, these new techniques will be implemented with minimal disruption to the existing software and maximal flexibility to use coprocessors efficiently.

The PI is a recognized leader in strongly coupled dark matter and both unsupervised and generative AI.
His leadership is supplemented by broad expertise in collider searches for new physics and scientific computing.
The program proposed here will advance multiple elements of the DOE HEP mission: understanding dark matter and improving capabilities in detector simulation.
The benefits will extend to the entire field and even into the future, beyond the HL-LHC, providing physics motivation and critical techniques for future colliders.
Success will be achieved by leveraging the PI's established collaborations based at Fermilab, which provides unparalleled and highly relevant resources and knowledge.

\subsection{Dark Matter}\label{subsec:dm}

\subsubsection{Background}\label{subsec:dmbkg}

Many astronomical observations, from various independent and complementary sources,
indicate that dark matter exists, comprises the majority of matter in the universe, and does not consist of any SM particles.
These sources include:
galaxy rotation curves, most of which do not match the expectation from visible matter~\cite{Rubin:1980zd,Persic:1995ru} except for a few~\cite{vanDokkum:2018vup,PinaMancera:2021wpc}, indicating that DM is unevenly distributed;
strong gravitational lensing from galaxy cluster collisions~\cite{Clowe:2006eq} and weak gravitational lensing from large-scale structures~\cite{Chang:2017kmv};
the cosmic microwave background power spectrum~\cite{Planck:2018vyg} and the matter power spectrum of the universe~\cite{Dodelson:2011qv,Planck:2018nkj};
and discrepancies in light element abundances from big bang nucleosynthesis~\cite{Pospelov:2010hj}.

Weakly interacting massive particles (WIMPs) have been explored for decades~\cite{Jungman:1995df}, with no direct evidence yet obtained for their existence.
Numerous searches have targeted their collider signature, an excess of events with large missing transverse momentum (\ptvecmiss, with magnitude \met),
including several of the most impactful led by the PI, motivated by hadronic supersymmetry~\cite{Khachatryan:2016kdk,Sirunyan:2017cwe,Sirunyan:2019hzr,Sirunyan:2019ctn}.
Alternative approaches, including the direct detection of interactions between dark matter and nuclei of visible matter
and indirect detection of dark matter annihilation or decay, similarly have not detected WIMP signatures.

Dark matter has been estimated from astrophysical measurements to have an abundance similar,
on the cosmological scale, to visible matter (roughly five times greater~\cite{Ade:2015xua}).
This correspondence implies that dark matter may consist of composite particles, much like the baryons that make up the majority of visible matter~\cite{Bai:2013xga},
possibly arising from a similar asymmetry mechanism~\cite{Petraki:2013wwa}.
Unlike WIMPs, exploration of composite dark matter models has only just started, and many such models have not been ruled out.

\subsubsection{Objectives}\label{subsec:dmobj}

The hidden valley models that produce strongly coupled dark matter include a new $SU(\Ncdark)$ strong force that binds dark quarks (\Pqdark) into dark hadrons, creating dark showers of new particles.
Depending on the details of the particles and forces in the hidden sector,
as well as the mediator particles that facilitate weak interactions between the SM and the hidden sector,
these dark showers can produce a variety of experimental signatures.
These include: \emph{semivisible jets} (SVJs), where \met from invisible dark hadrons is aligned with visible SM hadrons~\cite{Cohen:2015toa};
\emph{emerging jets} (EMJs), with multiple displaced vertices from decays of long-lived dark hadrons~\cite{Schwaller:2015gea};
and \emph{soft unclustered energy patterns} (SUEPs), in which unsuppressed large angle emissions lead to a spherical pattern of dark hadrons~\cite{Knapen:2016hky}.
These signatures are quite distinct from the expectations from WIMP dark matter and therefore are not probed by WIMP searches.

\subsubsection{Team and Collaborators}\label{subsec:dmteam}

\subsection{Detector Simulation}\label{subsec:sim}

\subsubsection{Background}\label{subsec:simbkg}

\subsubsection{Objectives}\label{subsec:simobj}

\subsubsection{Team and Collaborators}\label{subsec:simteam}

