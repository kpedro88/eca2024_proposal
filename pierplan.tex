The PI has a successful record of engaging members of underrepresented groups in research,
providing them with new opportunities to build their careers.
He feels a keen responsibility to strive for equity when conducting publicly-supported research
and will continue and expand these efforts with the early career funding.

\subsection{Researchers supported by this proposal}

In particular, the PI has served on numerous Fermilab hiring committees for a range of scientific and technical positions
and is trained to minimize bias when assessing and interviewing candidates.
As part of the Fermilab AI Project Office, he helped pioneer a new approach to hiring for the first groups of AI associates.
Minority serving institutions and identity and affinity groups were directly contacted to encourage their members to apply.
This procedure effectively reaches a much larger and more diverse audience than merely posting job openings in usual fora.
The resulting classes of AI associates have included members of underrepresented groups,
who have published impactful papers---such as Ref.~\cite{Ciprijanovic:2023hrw} by the PI,
AI associate A. Lewis, and their colleagues---and have gone on to subsequent positions at the lab or to graduate programs.

The PI will follow this procedure for all hiring supported by the early career funds,
including both postdoctoral researchers and AI associates.
This emphasis on diversity in recruiting will be bolstered by strong mentoring
in an equitable and accessible research environment, discussed below.
The budget for this project includes dedicated funds for supported researchers to attend conferences,
an important component of their career development.

\subsection{Collaboration with students}

The new research program proposed here incorporates several novel projects
perfectly suited for undergraduate, masters, and PhD students of various backgrounds.
Concretely, these tasks include studying the phenomenological behavior of dark QCD models;
analysis work such as trigger efficiency measurements and monitoring, background estimation, and systematic uncertainty estimation;
AI model development, optimization, and deployment; and computing performance studies.
These projects divide the work in the overall proposal into reasonable time segments of 1--2 years,
naturally producing results that will be useful for the students' career advancement.

The PI has supervised several undergraduate interns through Fermilab's SULI (Science Undergraduate Laboratory Internships) program,
as well as reviewing applications for the TARGET high school internship program.
These programs provide a diverse group who may not otherwise have access to national laboratory facilities and expertise.
Students supervised by the PI have gone on to graduate programs at prestigious institutes including CalTech, Stanford, and Cornell.
The PI is also joining the US CMS Collaboration's PURSUE (Program for Undergraduate Research SUmmer Experience) program as an advisor and mentor.
The technical research in this proposal directly builds on the first application of denoising to detector simulation in Ref.~\cite{Banerjee:2022gkg},
published by the PI and his interns.
The PI will continue to supervise Fermilab interns, also engaging with the CCI (Community College Internships) and SIST (Summer Internships in Science \& Technology)
that specifically target underrepresented groups.

He also has a long history of working with students from Central and South America,
primarily via the University of Puerto Rico, Mayaguez, which is a primarily undergraduate institution with no physics PhD program.
He has supervised undergraduate and masters students, some of whom participated in the above work,
while others joined him in the recent emerging jet search~\cite{CMS:2024gxp}.
He has recently strengthened this connection by co-directing
the second COFI school for advanced instrumentation and analysis techniques in Old San Juan, Puerto Rico.
This school brought together students from all backgrounds and multiple regions (North America, Europe, the Caribbean, and Central America)
to learn from top professors and scientists across physics, computing, mathematics, and AI, many recruited by the PI.
With early career support, the PI will be able to further expand his partnerships in this region,
providing many students with opportunities and guidance.

\subsection{Research environment}

Given the international nature of collider physics,
the PI is well versed in collaborating with people from many different cultures.
He is a strong advocate for respectful and inclusive conduct,
previously serving on a Fermilab focus group to improve and bring attention to the lab's community standards,
based on work undertaken by the CMS collaboration to succinctly summarize the important points of its code of conduct.
He reaches out to those interested in research and accommodates their needs,
which has led directly to the growth and success of the CMS Dark QCD team.

He particularly cares about ensuring and improving accessibility via technology.
For example, he promoted a color scheme for figures to help those with color vision deficiencies~\cite{Petroff:2021},
which is used in most figures in this proposal and is now officially recommended by the CMS collaboration.
He also widely promotes policies to post material in advance of meetings,
in case participants need to use screen readers or other accessibility aids.
He often reminds others to be considerate of time zones across the world,
including Asian time zones, which are frequently ignored in scheduling.

The PI will run his early career-supported team according to the Fermilab community standards~\cite{CommunityStandards} and the CMS code of conduct~\cite{CodeOfConduct}.
The Fermilab ``Integrity Counts'' system~\cite{ConcernsReporting} will be emphasized as an avenue to report any concerns or incidents.
In particular, we will make sure that the availability of this resource is communicated clearly
to users and affiliates, such as interns, students, and LHC Physics Center collaborators.
As part of our mentoring practices (Section~\ref{subsec:mentor}),
we will actively and frequently discuss the team's working environment and address any concerns promptly.
The principles of accessibility will also be emphasized.

\subsection{Mentoring and education}\label{subsec:mentor}

The PI has already guided several postdoctoral researchers at Fermilab
under the auspices of the CMS group's mentoring program,
which is so successful that its practices have spread to other departments at the lab.
He also mentors the lab's AI associates through regular biweekly meetings and informal interactions.
The PI will ensure these practices are followed in his own group, including:
writing and critiquing research plans, regular reports on research progress, practicing important presentations,
and developing strategies for career advancement (e.g. targeting specific conferences, nominating/applying for awards,
establishing relationships with senior scientists and professors for future recommendation letters).
We will also check in regularly with all personnel to ensure they are satisfied with their progress.

Through these activities, he has learned how to advise younger researchers to have successful careers.
He ensures that they take on impactful projects with clear paths to publication
and that they are identified as major contributors and leaders, e.g. via conference presentations.
He also ensures that they do not take on too much work and that they are making steady progress,
both with deliverables and with their own understanding and skills.
When advising students, he meets with them frequently, sometimes even on a daily basis depending on their needs,
and is always available to answer questions and help with problems.

With the LHC Physics Center, the PI has led numerous tutorials and exercises in the winter Data Analysis Schools
and the summer Hands-On Advanced Tutorial Sessions.
He has developed lectures and exercises to educate students on topics related to physics, computing, and AI,
which are often not taught well or at all in standard undergraduate curricula.
The students who work with the PI will benefit from these resources
and the PI's experience with these topics.

\subsection{Outreach and advocacy}

The PI is also an elected representative on the Fermilab Users Executive Committee,
where he chairs the Government Relations Subcommittee.
He promotes diversity and inclusion by expanding public informational material
and conveying the importance of these topics to Congress.
He is also a strong advocate for quality of life issues within the Fermilab community,
including those that affect access to laboratory facilities, temporary housing, and social connections,
because negative changes in these areas disproportionately impact people from disadvantaged groups.
