The PI has a successful record of engaging members of underrepresented groups in research,
providing them with new opportunities to build their careers.
He feels a keen responsibility to make equitable use of public research funding
and will continue and expand these efforts with the early career funding.
The corresponding new research program will produce many novel projects
perfectly suited for undergraduate, masters, and PhD students of all identities.
Concretely, these tasks include dark QCD model building;
analysis tasks such as trigger efficiency measurement and systematic uncertainty estimation;
AI model development and optimization; and computing performance studies.

The PI has supervised several undergraduate interns through Fermilab's SULI and CCI programs,
as well as guiding the TARGET high school internship program.
These programs provide a diverse group who may not otherwise have access to national laboratory facilities and expertise.
Students supervised by the PI have gone on to graduate programs at prestigious institutes including CalTech, Stanford, and Cornell.
The PI is also joining the US CMS Collaboration's PURSUE program as an advisor and mentor.
The technical research in this proposal directly builds on the first application of denoising to detector simulation in Ref.~\cite{Banerjee:2022gkg},
published by the PI and his interns.

He also has a long history of working with students from Central and South America,
primarily via the University of Puerto Rico, Mayaguez, which is a primarily undergraduate institution with no physics PhD program.
He has mentored undergraduate and masters students, some of whom participated in the above work,
while others joined him in the recent emerging jet search~\cite{CMS:2024emj}.
He has recently strengthened this connection by co-directing
the second COFI school for advanced instrumentation and analysis techniques in Old San Juan, Puerto Rico.
This school brought together students from all backgrounds and multiple regions (North America, Europe, the Caribbean, and Central America)
to learn from top professors and scientists across physics, computing, mathematics, and AI, many recruited by the PI.
With early career support, the PI will be able to further expand his partnerships in this region,
providing many students with opportunities and guidance.

Given the international nature of collider physics,
the PI is well versed in collaborating with people from many different cultures.
He is a strong advocate for respectful and inclusive conduct,
previously serving on a Fermilab focus group to improve and bring attention to the lab's community standards,
based on work undertaken by the CMS collaboration to succinctly summarize the important points.
He reaches out to those interested in research and accommodates their needs,
which has led directly to the growth and success of the CMS Dark QCD team.
He particularly cares about accessibility;
for example, he promoted a color scheme for figures to help those with color vision deficiencies~\cite{Petroff:2021},
which is now being officially adopted by the entire CMS collaboration.
He also widely promotes policies to post material in advance of meetings,
in case participants need to use screen readers or other accessibility aids.
He often reminds others to be considerate of time zones across the world,
including Asian time zones, which are frequently ignored in scheduling.

The PI has also guided several postdoctoral researchers at Fermilab,
under the auspices of the CMS group's mentoring program,
which is so successful that its practices have spread to other departments at the lab.
Through this, he has learned how to advise younger researchers to have successful careers.
He ensures that they take on impactful projects with clear paths to publication
and that they are identified as major contributors and leaders, e.g. via conference presentations.
He also ensures that they do not take on too much work and that they are making steady progress,
both with deliverables and with their own understanding and skills.
When advising students, he meets with them frequently, sometimes even on a daily basis,
and is always available to answer questions and help with problems.

Through the LHC Physics Center, the PI has led numerous tutorials and exercises in the winter Data Analysis Schools
and the summer Hands-On Advanced Tutorial Sessions.
He has developed lectures and exercises to educate students on topics related to physics, computing, and AI,
which are often not taught well or at all in standard undergraduate curricula.
The students who work with the PI will benefit from these resources
and the PI's experience with these topics.

The PI is also an elected representative on the Fermilab Users Executive Committee,
where he chairs the Government Relations Subcommittee.
He promotes diversity and inclusion by expanding public informational material
and conveying the importance of these topics to Congress.
He is also a strong advocate for quality of life issues within the Fermilab community,
including those that affect access to laboratory facilities, temporary housing, and social connections,
because negative changes in these areas disproportionately impact people from disadvantaged groups.
