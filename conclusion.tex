\section{Conclusion}\label{sec:conclusion}

This proposal puts forth a novel program to uncover the origins of dark matter
by searching for evidence of strongly coupled dark sectors or ``dark QCD'' (quantum chromodynamics)
using the Compact Muon Solenoid (CMS) experiment at the Large Hadron Collider (LHC).
The program is highly complementary to other ongoing efforts:
direct detection and annihilation experiments have highly suppressed sensitivity to these models,
and they produce phenomena overlooked in previous collider searches.
The new signatures include semivisible jets, emerging jets, and soft unclustered energy patterns.
Building on the initial explorations using the LHC Run 2 dataset,
we will unify the trigger, search, and jet identification strategies,
focusing on semivisible jets as the most subtle signature.
The latest unsupervised artificial intelligence (AI) techniques will be employed throughout to maximize model independence,
even as we continue to build more thorough models of the behavior of dark QCD.
The search will be executed using the LHC Run 3 dataset,
to benefit from the increases in center-of-mass energy and integrated luminosity.
Finally, we will scan combinations of all dark QCD phenomena to identify unconstrained signatures
and design the next steps in the program.

To facilitate this program as well as many other LHC searches and measurements,
we will develop fast and accurate detector simulation using generative AI.
Using cutting-edge diffusion models and applying the latest acceleration techniques,
we will generate particle showers hundreds of times faster than CPU-based full detector simulation software.
The algorithms will be deployed to coprocessors across the computing grid using inference as a service,
the most efficient and portable approach.
In addition to enabling the dark QCD scan over the entire model space,
AI-based simulation will resolve a major computing challenge for the high luminosity LHC upgrade.
The same techniques, in combination with new GPU-based classical simulation engines,
will push forward the design of the future muon collider and its detector systems
by simulating the beam-induced background (BIB) from muon decays in flight, which is currently infeasible.

The PI is uniquely suited to deliver both aspects of the proposal.
His expertise and leadership in dark QCD, AI algorithms, detector simulation, inference on coprocessors, and experiment software are widely acknowledged.
As a Fermilab scientist, he can leverage the lab's computational and theory expertise and direct the lab's facilities to ensure success.
The results will significantly advance the mission of the DOE Office of Science to search for physics beyond the standard model and to make optimal use of national computing resources.

\subsection{Timetable of Activities and Deliverables}

Table~\ref{tab:activities} summarizes the milestones described in the previous sections.
The immediately achievable parts of each project are scheduled first,
while building the tools and expertise needed to handle the next steps.
The last two budget years are devoted to the final, most ambitious deliverables.
BY1 and BY2 correspond to the remaining years of LHC Run 3, while BY5 corresponds to the final year of the long shutdown before HL-LHC operations commence.

The efforts of the PI and the RA in leading the Run 3 dark QCD searches, along with collaborators from the CMS Dark QCD team,
will result in new dark QCD models in BY2, at least one limited-author AI algorithm publication and the Run 3 semivisible jet search publication in BY3, and the full dark QCD scan publication in BY5.
The PI, the RA, and the AI associates will deliver the AI diffusion-based simulation for the CMS detector in BY3
and for the muon collider BIB in BY5, with corresponding technical publications.

\begin{table}[!hbtp]
\vspace{\myfigurespacing}
\begin{center}
\begin{tabular}{|c|c|c|c|c|c|}
\hline
Deliverable & BY1 & BY2 & BY3 & BY4 & BY5 \\
\hline
Model building & \cellcolor{blue!25} & \cellcolor{blue!25} & & & \\
\hline
Anomaly trigger commissioning & \cellcolor{blue!25} & \cellcolor{blue!25} & & & \\
\hline
Tagger and JVN development & \cellcolor{blue!25} & \cellcolor{blue!25} & \cellcolor{blue!25} & & \\
\hline
Run 3 semivisible jet search & & \cellcolor{blue!50} & \cellcolor{blue!50} & & \\
\hline
Dark QCD scan & & & \cellcolor{blue!75} & \cellcolor{blue!75} & \cellcolor{blue!75} \\
\hline
\hline
Diffusion model development & \cellcolor{orange!25} & \cellcolor{orange!25} & \cellcolor{orange!25} & & \\
\hline
Fast simulation refinement & \cellcolor{orange!25} & \cellcolor{orange!25} & \cellcolor{orange!25} & & \\
\hline
Inference as a service for simulation & & \cellcolor{orange!50} & \cellcolor{orange!50} & \cellcolor{orange!50} & \\
\hline
Future collider simulation & & & & \cellcolor{orange!75} & \cellcolor{orange!75} \\
\hline
\end{tabular}
\vspace{\myfigureskip}
\caption{A summary of the activities and deliverables for the strongly coupled dark matter searches and fast detector simulation development in each budget year.}
\label{tab:activities}
\end{center}
\end{table}

\clearpage
