All data products from the published results of the dark QCD searches in this proposal will be recorded in the HEPData database~\cite{Maguire:2017ypu}.
In addition, supplementary information will be provided via the CMS Publications online repository~\cite{cmspub}.
(Ref.~\cite{CMS:2021dzg} is already available at these locations~\cite{hepdata,EXO-19-020}.)
The observed and simulated data used to derive those results are recorded and produced by the CMS collaboration,
and they are handled according to the CMS data preservation policy~\cite{cmsdata}.
These data are eventually released for public use, as described in Ref.~\cite{cmsdata};
the public datasets are hosted in the CERN OpenData Portal~\cite{opendata}, stored in the ROOT format.
Both HEPData and CERN OpenData follow the FAIR (Findable, Accessible, Interoperable, and Reusable) principles.

The PI, as CMS ML4Sim convener, will pursue the publication of the simulated datasets used for training and testing the diffusion algorithm,
following the example set by the ATLAS collaboration in the public \challenge.
These datasets will be hosted on Zenodo, also following FAIR principles.
Other training datasets will be published as allowed by the CMS collaboration.

The code and implementation details for most existing AI and computational algorithms included in this proposal are provided in public code repositories:
AXOL1TL~\cite{AXOL1TL:repo}, CICADA~\cite{CICADA:repo}, EVN~\cite{EVN:repo}, diffusion~\cite{CaloDiffusion:repo}, refinement~\cite{Refinement:repo}, and SONIC~\cite{SONIC:repo}.
The WNAE algorithm is in the process of being published and public code will be provided once the publication is submitted.
Public code will provided for updates to these algorithms and new algorithms, such as the JVN and the WNGAE.
These repositories have adopted permissive open-source licensing, following the CMS collaboration's decision to publish its software under the Apache 2.0 license.
As a former CMS software release manager and a frequent instructor of Git version control tutorials for CMS researchers,
the PI will ensure that best practices for open, collaborative software development are followed for all software products in this proposal.
These include code reviews, continuous integration and testing, documentation, and the usage of containers for reproducibility.
